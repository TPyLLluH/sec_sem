\begin{flushleft}
    \setlength{\parindent}{14pt}
    \hspace*{10pt}
    \begin{center}
        \textbf{Введение}
    \end{center}
    \vspace*{14pt}

    Использование информационных технологий в настоящее время невозможно при отсутствии рациональной
организации данных, а также предоставления эффективного доступа к ней пользователей.
База данных "--- это упорядоченный набор структурированной информации либо данных,
которые как правило хранятся в электронном виде во компьютерной системе. \\

Системы баз данных на сегодняшний день считаются основой построения большинства информационных
систем и применяются при автоматизации практически всех сфер человеческой деятельности.
К примеру, доступ к базе данных нужен при работе с библиотечной информационной
системой, включающей данные об абсолютно всех книгах, существующих в библиотеке, ее читателях,
заявках на бронирование книг и т.д. В ней как правило присутствуют средства, разрешающие
читателям находить необходимую им книгу согласно названию, фамилиям авторов либо указанной
тематике. С поддержкой подобного рода систем организуется учет перемещения книг, другие операции,
требуемые в библиотечной деятельности.\\

Данные в наиболее известных типах современных баз данных обычно
хранятся в виде строк и столбцов создающих таблицу. Этими сведениями можно
легко управлять, менять, обновлять, контролировать и упорядочивать [1].\\
На сегодняшний день ни одна организация не способна обойтись без баз данных, так
как она повышает оперативность работы со информацией, поэтому с каждым годом растет востребованность 
разработчиков баз данных.
    
    \newpage
    \begin{center}
        \textbf{Виды баз данных}
    \end{center}
    \vspace*{14pt}

    На сегодняшний день существует довольно много видов баз данных(более 10),
    но в основном используются следующие:
    \begin{itemize}
        \item Реляционные;

        \item Иерархические
        
        \item Сетевые.
    \end{itemize}

Реляционные базы данных.\\
В реляционной базе данных вся информации представляется в виде таблиц, 
и любые операции над данными "--- это операции над таблицами. Таблицы строят 
из строк и столбцов. Строки "--- это записи, а столбцы представляют собой 
структуру записи (каждый столбец имеет определенный тип данных и длину данных). 
Строки в таблице не упорядочены "--- не существует первой или десятой строки. 
Однако поскольку на строки необходимо как"=то ссылаться, то вводится понятие «первичный ключ».[2]\\

Первичный ключ "--- это столбец, значение которого во всех строках разные. Используя 
первичный ключ, можно однозначно ссылаться на любую строку таблицы.\\

Иерархические базы данных.\\
Иерархические базы данных "--- это самая первая модель представления данных,
в которой все записи базы данных представлены в виде дерева, с соотношением предок"=потомок.
Фактически данные отношения реализуются в виде указателей на предков и потомков, содержащихся в самой записи.
Однако иерархическая модель не является оптимальной, так как в такой базе данных может 
информация может дублироваться(Например, допустим, что один и тот же тип болтов используется в 
автомобиле 300 раз в различных узлах. При использовании иерархической модели, данный 
тип болтов будет фигурировать в базе данных не 1 раз, а 300 раз (в каждом узле "--- отдельно)).\\

Чтобы устранить этот недостаток была введена сетевая модель представления данных.

Сетевые базы данных.\\
Сетевая база данных "--- это база данных, в которой одна запись может участвовать в нескольких отношениях предок-потомок . 
Т.е. фактически, база данных представляет собой не дерево, а граф.
Физически данная модель также реализуется за счет хранящихся внутри самой записи указателей на другие записи, только, 
в отличие от иерархической модели, число этих указателей может быть произвольным.

И иерархическая и сетевая модели достаточно просты, однако они имеют общий недостаток: для того, 
чтобы получить ответ даже на простой вопрос, программист должен был написать программу, которая 
просматривала базу данных, двигаясь по указателям от одной записи к другой. Написание программы 
занимало некоторое время, и часто к тому моменту, когда такая программа была написана, необходимость 
в получении данных уже не требовалась. Поэтому произошел практически повсеместный переход к реляционным базам данных[3].

\newpage
    \begin{center}
        \textbf{Базы данных и этапы их разработки. Система управления базами данных(СУБД)}
    \end{center}
    \vspace*{14pt}

    Итак, база данных "--- это хранилище информации.\\
    Система управления базами данных(СУБД) "--- это программный комплекс, который позволяет 
    администрировать базу, защищает ее целостность и конфиденциальность сведений[4].\\
    Системы управления бывают разными: различаются типы баз данных, особенности представления 
    информации внутри базы, методы управления и языки, на которых пишутся запросы. \\
    Для обращения к БД чаще всего используется язык структурированных запросов SQL. 
    Если пользователю необходимо прочитать данные из базы данных, он запрашивает их у
    СУБД с помощью SQL. СУБД обрабатывает запрос, находит требуемые данные и посылает их пользователю.
    Примеры популярных СУБД: MySQL, PostgreSQL, Oracle Database[5].


    Разработка базы данных разбивается на следующие основные этапы:\\
    \begin{enumerate}
        \item Постановка задачи. \\
        На этом этапе определяются цели разработки: что должно получиться 
        в результате. При этом следует получить ответы на множество вопросов:
        \begin{itemize}
            \item Сколько примерно человек должны пользоваться базой?
            \item Примерные объемы информации
            \item Как часто появляются и изменяются данные?
            \item Будет ли система развиваться в дальнейшем?
            \item Должна ли она быть автономной или являться частью другой информационной системы?
            \item Каковы требования к защите информации от посторонних?
            \item Насколько серьезной должна быть защита от сбоев?
            \item Каковы требования по скорости доступа к информации?
            \item Какого рода информация должна храниться?\\
            и т.д.
        \end{itemize}
        Детальная проработка требований к системе на этом этапе позволит более точно определить 
        и сроки, и стоимость работ, и принимать обоснованные решения на следующих этапах.
        \item Разработка информационно"=логической модели. \\
        Проводится детальное обследование предметной области. Определяется 
        перечень входной и выходной информации и детальные характеристики 
        этой информации. Выявляются связи между отдельными объектами предметной 
        области. Построенная в результате модель представляет собой информационную
        картину решаемой задачи. При этом еще не решаются технические вопросы 
        выбора оборудования, СУБД и т.п.
        
        \item Выбор СУБД. Разработка логической модели базы данных. \\
        Опираясь на результаты первого и второго этапов, принимается решение об
        используемой СУБД. На основе инфологической модели создается детальное 
        описание данных в терминах выбранной СУБД (логическая модель). На этом 
        этапе производится распределение данных по таблицам, описывается структура
        каждой таблицы (состав и характеристики полей, ключи, индексы, связи и т.п.).\\

        Если выбранная СУБД по каким"=то параметрам не подходит, то производится 
        или изменение требований к системе, или выбирается другая СУБД.
        
        \item Разработка программного обеспечения базы данных. \\
        Созданные на 3"=м этапе таблицы заполняются данными контрольного примера. 
        Разрабатываются дополнительные объекты базы данных: запросы, программные 
        модули, формы для работы с данными, печатаемые на основе данных базы 
        документы и т.п. Результаты разработки проверяются на контрольном примере. 
        Желательно согласовывать результаты с персоналом, который в будущем будет работать с базой.\\
        Составляются описания, как для будущих администраторов базы, так и для пользователей.
        \item Заполнение базы рабочими данными и поддержание ее в актуальном состоянии.
        Производится первичное обучение пользователей. Вводятся необходимые для дальнейшей работы данные. 
        Разрабатываются и внедряются организационные документы, закрепляющие обязанности персонала при 
        работе с базой. Выполняются необходимые доработки по вопросам, выявившимся в процессе эксплуатации[6].
    \end{enumerate}\
    

\newpage
    \begin{center}
        \textbf{Что должен знать и уметь разработчик баз данных?}
    \end{center}
    \vspace*{14pt}

    Разработчик баз данных (Database Developer или Database Programmer) отвечает за создание, 
    администрирование и устранение неполадок компьютерных баз данных, которые могут обрабатывать 
    большие объемы информации и обеспечивать ее безопасность[7]. В его обязанности входит 
    использование кода и веб"=архитектуры для проектирования систем данных, анализа и поддержки 
    существующих баз данных и внедрения новых пользовательских функций. Роль разработчика 
    БД состоит в том, чтобы максимально облегчить пользователям баз данных доступ к необходимой
    им информации и сохранить информационные системы для будущих разработок компании.\\

    Итак, что делает разработчик баз данных:
    \begin{itemize}
        \item проектирует базы данных 
        (выбирает подходящие инструменты, языки програмирования, анализирует потребности системы компании и т. д.);
        \item обеспечивает эффективное использование созданной базы данных;
        \item налаживает и сопровождает БД;
        \item обеспечивает безопасность данных, которые хранятся на сервере;
        \item анализирует жалобы и предложения пользователей, устраняет ошибки;
        \item консультирует системных администраторов;
        \item анализирует нагрузки и обновление ядра СУБД;
        \item работает с администраторами, программистами и архитекторами баз данных.
    \end{itemize}
Разработчику баз данных могут заказывать модернизацию и сопровождение 
уже имеющейся БД, чтобы повысить ее производительность и безопасность. 
Если в штате компании нет системного администратора, дополнительной 
обязанностью Database Developer’а может стать работа с коллективом: 
консультации по БД, обучение. \\

Работа с данными "--- это огромная ответственность. Если сбой в системе или ошибка 
разработчика станет причиной потери информации, то владелец данных может понести 
колоссальные убытки. Database Developer должен великолепно знать свою работу, уметь 
быстро устранять ошибки, чтобы не допускать утери или утечки информации.

\newpage
    \begin{center}
        \textbf{Заключение}
    \end{center}
    \vspace*{14pt}

    Основу большинства   информационных   технологий составляют   
    большие массивы   накопленной   информации.   Основной   формой   организации   
    хранения данных   в   информационных   системах   являются   базы   данных.
    Обработка баз данных всегда была важной темой, целью которой было помочь людям
    вести учет данных[8].\\
    Роль разработчика БД состоит в том, чтобы максимально 
    облегчить пользователям баз данных доступ к необходимой им информации и 
    сохранить информационные системы для будущих разработок компании.
    В его обязанности входит использование кода и веб"=архитектуры для проектирования 
    систем данных, анализа и поддержки существующих баз данных и внедрения новых 
    пользовательских функций.
    Поэтому сейчас разработчик баз данных "--- одна из самых востребованных специальностей.

    \newpage
    \begin{center}
        \textbf{Список использованных источников}
    \end{center}
    \vspace*{14pt}

    \begin{enumerate}
        \item База данных "--- определение, типы баз данных https://www.oracle.com/cis/database/what-is-database/
        \item Реляционные базы данных https://oracle-patches.com/db/реляционные-базы-данных-объяснение-понятий,-вводный-обзор
        \item Иерархические и сетевые базы данных https://studopedia.su/10\_88967\_lektsiya--bazi-dannih--teoreticheskiy-obzor.html
        \item Определение СУБД - https://blog.skillfactory.ru/glossary/subd/
        \item Популярные системы управления базами данных https://blog.skillfactory.ru/glossary/baza-dannyh/
        \item Основные этапы разработки базы данных https://studopedia.ru/19\_302231\_osnovnie-etapi-sozdaniya-bazi-dannih.html
        \item Профессия разработчик баз данных https://www.profguide.io/professions/database\_developer.html
        % \item Чем занимается разработчик баз данных https://abcdwork.ru/professii/razrabotchik-baz-dannyx.html
        \item Роль разработчика баз данных https://kinobaza24.ru/biography/razrabotchik-baz-dannyh-navyki.html
    \end{enumerate}

\end{flushleft}

