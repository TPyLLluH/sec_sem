\begin{flushleft}
    \setlength{\parindent}{14pt}
    \hspace*{10pt}
    \begin{center}
        \textbf{Введение}
    \end{center}
    \vspace*{14pt}

    Применение информационных технологий в настоящее время невозможно без рациональной 
    организации информации и обеспечения эффективного доступа к ней пользователей    
    База данных — это упорядоченный набор структурированной информации или данных, 
    которые обычно хранятся в электронном виде в компьютерной системе. \\

    Системы баз данных сегодня являются основой построения большинства информационных 
    систем и используются при автоматизации практически всех сфер человеческой деятельности. 
    Например, доступ к базе данных необходим при работе с библиотечной информационной
    системой, содержащей сведения обо всех книгах, имеющихся в библиотеке, ее читателях, 
    заявках на бронирование книг и т.д. В ней обычно содержатся средства, позволяющие 
    читателям находить нужную им книгу по названию, фамилиям авторов или указанной 
    тематике. С помощью такого рода систем организуется учет движения книг, другие операции, 
    необходимые в библиотечной деятельности.\\

    Данные в наиболее распространенных типах современных баз данных обычно 
    хранятся в виде строк и столбцов формирующих таблицу. Этими данными можно 
    легко управлять, изменять, обновлять, контролировать и упорядочивать.\\


\end{flushleft}

