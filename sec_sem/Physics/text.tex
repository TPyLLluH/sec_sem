\begin{flushleft}
    \setlength{\parindent}{14pt}
    \hspace*{10pt}
    \textbf{Введение}
    \vspace*{14pt}

    Материя имеет сложную структуру. Основываясь на достижениях современной науки 
    мы можем указать некоторые из её типов и структурных уровней. Известно, что к концу XIX в. 
    естественная наука не выходила за рамки молекул и атомов. С открытием электронной радиоактивности 
    начался прорыв физики в более глубокие области материи. В настоящее время физика открыла 
    много разных элементарных частиц. Оказалось, что у каждой частицы есть свой антипод "--- античастица,
    с той же массой, но противоположным зарядом и т.д. Нейтральные частицы также имеют 
    античастицы. Частицы и античастицы, взаимодействуя, <<уничтожаются>>, т.е. исчезают, превращаются 
    в другие частицы. Например, электрон и позитрон, разрушаясь, превращаются в два фотона. 
    Симметрия элементарных частиц позволяет нам сделать предположение о возможности существования 
    антимира, состоящего из античастиц, антиатомов и антивещества. И все законы, действующие в антимире,
    должны быть похожи на законы нашего мира. 

    \newpage
    \textbf{Что такое антиматерия?}
    \vspace*{14pt}
    
    Объекты Вселенной "--- галактики, звезды, квазары, планеты, сверхновые, 
    животные и люди "--- состоят из материи. Они образованы различными элементарными 
    частицами "--- кварками, лептонами, бозонами. Однако оказалось, что существуют частицы, 
    у которых одна часть свойств полностью соответствует параметрам <<оригиналов>>, а 
    другая имеет обратные значения. Это свойство побудило ученых дать таким частицам общее название <<антивещество>>. \\

Судя по имеющимся сегодня данным, нет никаких антигалактик, антизвезд или других крупных 
объектов из антивещества. И это очень странно: согласно теории Большого взрыва, в момент 
возникновения нашей вселенной появилось одинаковое количество материи и антивещества, и куда 
это делось, неясно. В настоящее время этому явлению есть два объяснения: либо антивещество исчезло 
сразу после взрыва, либо оно существует в некоторых отдаленных частях вселенной, и мы просто 
еще не обнаружили его. Такая асимметрия является одной из важнейших нерешенных проблем современной физики.\\

Антивещество "--- это вещество, состоящее из античастиц "--- <<зеркальных отражений>> ряда элементарных
 частиц, которые имеют одинаковый спин и массу, но отличаются друг от друга знаками всех других 
 характеристик взаимодействия: электрический и цветового заряд, барионные и лептонные квантовые числа. 
 Некоторые частицы, например фотон, не имеют античастиц или являются эквивалентными античастицами по отношению к себе. \\
 
 Как сегодня считается, античастицы реагируют на фундаментальные силы, определяющие структуру материи 
 (сильное взаимодействие, образующее ядра, и электромагнитное, образующее атомы и молекулы), совершенно 
 одинаково, поэтому структура антивещества должна быть такой же, как структура «нормального» вещества.

 \newpage
 \textbf{Чем отличается вещество от антивещества?}
 \vspace*{14pt}

 Что касается антивещества "--- к ним относятся аналоги элементарных частиц с противоположным зарядом, 
 магнитным моментом и некоторыми другими свойствами. Конечно, все свойства частицы не могут измениться 
 на противоположные. Например, масса и время жизни всегда должны оставаться положительными, 
 сосредоточившись на них, можно отнести частицы к одной категории (например, протоны или нейтроны). \\

Если мы сравним протон и антипротон, некоторые свойства будут одинаковыми: масса у обоих составляет 938.2719 (98) 
мегаэлектронвольт, спин $1/2$. Но электрический заряд протона равен 1, а у антипротона минус 1, барионное число 
(оно определяет количество сильно взаимодействующих частиц, состоящих из трех кварков) равно 1 и минус 1 соответственно.\\

Некоторые частицы, такие как бозон Хиггса и фотон, не имеют антианалогов и называются истинно нейтральными.\\

Большинство античастиц появляются вместе с частицами в процессе, называемом <<рождением пар>>. Для образования 
такой пары требуется высокая энергия, то есть огромная скорость. В природе античастицы возникают при столкновении 
космических лучей с атмосферой Земли в массивных звездах, расположенных вблизи пульсаров и активных ядер 
галактик. Ученые также используют для этого ускорительные коллайдеры, собирая облака антипротонов после столкновения 
пучка протонов с металлической мишенью и аккуратного замедления разлетающихся частиц, чтобы их можно было использовать 
в последующих экспериментах. \\


\newpage
\textbf{Взаимодействие вещества и антивещества. Почему антивещество сложно получить?}
\vspace*{14pt}

Важнейшим фактором взаимодействия частицы с ее античастицей является их общая аннигиляция, 
при которой либо выделяются высокоэнергетичные фотоны, либо возможно появление иной пары 
частица"=античастица. В любом случае, выделившаяся энергия описывается знаменитой формулой Альберта Эйнштейна:
\begin{equation*}
    E = mc^2
\end{equation*} \\
Уничтожение всего 1 грамма вещества и антивещества приводит к высвобождению $10^{14}$ Дж энергии, что 
 эквивалентно взрыву средней атомной бомбы в 10 килотонн. \\

Изучать это вещество гораздо сложнее, 
чем зарегистрировать. В природе античастицы в стабильном состоянии еще не
встречались. Проблема в том, что материя и антиматерия
уничтожают друг друга при <<контакте>>.
Получить антивещество в лабораториях вполне возможно— но его довольно сложно сохранить. 
До сих пор ученым удавалось делать это всего несколько минут.\\

Проблема хранения антивещества вызывает у физиков настоящую головную боль, потому что антипротоны
и позитроны мгновенно разрушаются, когда сталкиваются с частицами обычной материи.
Чтобы сохранить их, ученым пришлось разработать умные устройства, которые могли бы предотвратить
это. \\

Заряженные частицы антивещества, такие как позитроны и антипротоны, могут храниться в так называемых ловушках Пеннинга.
 Они похожи на крошечные ускорители частиц. В них частицы движутся по спирали до тех пор, пока магнитные и электрические 
поля удерживают их от столкновения со стенками ловушки.\\

Однако ловушки Пеннинга не работают для нейтральных частиц, таких как антиводород. Поскольку у них нет заряда, эти частицы
вы не можете ограничить электрическими полями. Они содержатся в ловушках Джоффе, которые работают, создавая область 
пространства, где магнитное поле увеличивается во всех направлениях. Частицы антивещества застревают в области
с самым слабым магнитным полем.\\

Магнитное поле Земли может действовать как ловушки антивещества. Антипротоны были обнаружены в определенных областях
вокруг Земли - радиационных поясах Ван Аллена.


\newpage
\textbf{Заключение}
\vspace*{14pt}

Антиматерия также лежит в основе ряда других экспериментов. Целые антиатомы производятся 
на Антипротонном замедлителе ЦЕРН, и они обеспечивают целый ряд экспериментов по проведению 
высокоточных измерений. Эксперимент AMS"=2 на борту Международной космической станции находится 
в поисках антиматерии космического происхождения. Ряд текущих и будущих экспериментов будет 
посвящен вопросу о том, существует ли асимметрия вещества"=антивещества среди нейтрино. \\
Всего горстка антиматерии может произвести огромное количество энергии. Это делает ее популярным 
топливом для футуристических транспортных средств в научной фантастике. Вообще ракетный двигатель 
на антивеществе гипотетически возможен; главное ограничение — это накопление достаточного количества 
антивещества, чтобы использовать его.\\

Хотя мы до сих пор не можем полностью разгадать тайну асимметрии материи и антиматерии, 
наше последнее открытие открыло дверь в эпоху точных измерений, которые могут раскрыть еще 
неизвестные явления.

\newpage
\textbf{Список использованной литературы}
\vspace*{14pt}

\begin{enumerate}
    \item Власов Н. А. “Антивещество” М.: 1960.
    \item https://naked-science.ru/article/astronomy/uchenye-predpolozhili-nalichie
    \item Альфвен Х. Миры и антимиры. Космология и антиматерия. М., 1968.
    \item https://hi-news.ru/science/pochemu-vo-vselennoj-bolshe-materii-chem-antimaterii.html
    \item https://mining-cryptocurrency.ru/antimateriya-antiveshchestvo-antichasticy/
\end{enumerate}
\end{flushleft}
