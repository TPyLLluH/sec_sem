\begin{flushleft}
    \setlength{\parindent}{14pt}
    \hspace*{10pt}
    \begin{center}
        \textbf{Введение}
    \end{center}
    \vspace*{14pt}

    Защита здоровья подрастающего поколения является важнейшей стратегической задачей государства, поскольку
    основой здоровья взрослого населения страны закладывается в детстве.
    Спорт социальные движения объединяют личные и общественные интересы,
    формируют здоровый морально"=психологический и нравственный климат в различных
    социально"=демографических группах населения, особенно среди молодежи. Все перспективы
    социально"=экономического развития государства, высокий уровень жизни населения, уровень развития
    науки и культуры "--- это результат хорошего здоровья, которого достигают дети сегодня. \\

    Следовательно, с целью дифференцированного подхода к организации занятий физкультурой,
    все студенты общеобразовательных учреждений, в зависимости от состояния их здоровья,
    делятся на три группы: основная, подготовительная и специальная медицинские группы.
    Классы в этих группах различаются по учебным планам, объему и структуре
    физической нагрузки, а также требованиями к уровню усвоения учебных материалов.

    \newpage
    \begin{center}
        \textbf{Характеристика специальной медицинской группы здоровья}
    \end{center}
    \vspace*{14pt}

Специальная медицинская группа делится на две: специальную <<А>> и специальную <<Б>>.
Окончательное решение об отправке студента в специальную медицинскую группу
производится врачом после дополнительного обследования. Учащиеся со специальной медицинской группой 
имеют  значительные отклонения в состоянии  здоровья постоянного или временного характера,
не мешающие выполнению обычной  учебной работы в школе, но являющиеся противопоказанием для 
занятий  физкультурой. Некоторые из них временно, впредь до улучшения
состояния здоровья, совсем освобождаются от участия  в учебных занятиях по физкультуре.\\

В специальную группу А (\MakeUppercase{\romannumeral 3} группа здоровья) входят учащиеся с
отклонениями в постоянном состоянии здоровья (хронические заболевания, врожденные
пороки развития в фазе компенсации) или временного характера или в физическом развитии,
не препятствующие осуществлению нормальной учебной или воспитательной работы, но требующие
ограничения на физическую активность. Отнесенным к этой группе разрешается
заниматься физкультурой в образовательных учреждениях только по специальным программам (оздоровительные
корректирующие и укрепляющие здоровье технологии), которые согласовываются с органами здравоохранения и
лицензированным директором под руководством учителя физкультуры или инструктора,
прошедшего специальные курсы повышения квалификации.\\

На уроках физкультуры обязательно учитывается характер и степень выражения
отклонения в состоянии здоровья, физическом развитии и функциональном уровне участника.
В этом случае упражнения на скорость, силу, акробатику очень строго ограничены; подвижные игры умеренной
интенсивности.\\

В специальную группу В (\MakeUppercase{\romannumeral 4} группа здоровья) входят учащиеся со значительными отклонениями
в состоянии постоянного здоровья (хронические заболевания на стадии субкомпенсации) и временно
характера, но без выраженных нарушений здоровья и допущен к теоретическим занятиям
в учебных заведениях.\\

Для этой группы рекомендуется в обязательном порядке посещать занятия по ЛФК на кафедрах
лечебной физической культуры местной поликлиники, лечебно"=спортивный диспансер.
Допустимо проводить регулярные самостоятельные занятия дома в предлагаемых комплексах
врача ЛФК. \\

Важно иметь в виду, что учащиеся, отнесенные по состоянию здоровья к специальным группам, нуждаются в двигательной активности не
меньше, а чаще всего больше, чем здоровые люди, причем таким учащимся
требуется качественно иная двигательная активность.



\newpage
\begin{center}
    \textbf{Комментарии к относящимся к специальной группе контрольным нормативам}
\end{center}
\vspace*{14pt}

С целью контроля уровня физической подготовленности учащихся на разных этапах обучения
министерством образования устанавливаются специальные контрольные нормативы.

Несколько нормативов, предназначенных для учащихся со специальной медицинской группой здоровья. \\

\begin{itemize}
    \item Пресс "--- 40(макс. 50) раз
    \item Гиперэкстензия "--- 40(макс. 50) раз
    \item Приседание "--- 30(макс. 50) раз
    \item Наклон вперед из положения сидя "--- +7"=10 см.
    \item Бег 1000 м. "--- 6 мин.
    \item Бег 2000 м. "--- без учета времени.
\end{itemize}
Все нормативы хотя и являются необходимыми, но могут быть упрощены в зависимости от состояния здоровья учащегося.\\

Нормативы — это официальное требование Министерства образования. 
Государство заинтересовано в том, чтобы последующее поколение 
было здоровым, поэтому и существуют нормы, которые должны мотивировать учащихся
укреплять свое здоровье и развивать физические качества.\\


\newpage
\begin{center}
    \textbf{Основные требования безопасности на месте занятий физической культурой}
\end{center}
\vspace*{14pt}

Приступая к занятиям, учащиеся обязаны ознакомиться с инструкцией, знать техни­ку выполнения 
элементов физических упражнений.\\
Ознакомление с требованиями по технике безопасности для всех учащихся должно быть обязательным. \\
Приступая к специализированным занятиям, учащиеся знакомятся с порядком и особенностями как страховки, 
так и взаимопомощи друг другу, так как осо­бенности выполнения тех или иных упражнений, связанных с конкретными 
умениями и навыками, предъявляет к ним тот необходимый минимум обязательных знаний, который позволяет избежать 
травматизма, а также быстрее освоить само упражнение.\\
Преподаватели физической культуры несут прямую ответственность за охрану жиз­ни и здоровья учащихся.
\vspace*{14pt}

Общие требования к учащимся:
\begin{itemize}
    \item Не отлучаться с места занятий без разрешения преподавателя;
    \item Спортивная одежда должна соответствовать времени года, температуре воздуха и виду занятий;
    \item Перед началом любого вида деятельности выполняется разминка;
    \item Почувствовав боль, недомогание в процессе занятия, занимающийся должен со­общить об этом преподавателю;
    \item Спортивный инвентарь и оборудование должны использоваться только по назна­чению.
\end{itemize}
\vspace*{14pt}

Легкая атлетика:
\begin{itemize}
    \item Все задания должны выполняться только после специальной разминки;
    \item Одежда и обувь должны соответствовать направленности занятия с учетом погод­ных условий;
\end{itemize}
\newpage

Бег:
\begin{itemize}
    \item Не разговаривать во время бега;
    \item Беговые задания выполнять только в указанном направлении, с целью избежать столкновения при беге;
    \item При недомогании во время бега обратиться к преподавателю.
\end{itemize}
\vspace*{14pt}

Метания:
\begin{itemize}
    \item Не находится в секторе для метания;
    \item Метания выполняются только в том направлении, которое указал преподаватель.
\end{itemize}
\vspace*{14pt}

Прыжки:
\begin{itemize}
    \item Прыжковые задания выполняются в строго отведенном месте;
    \item Запрещается находиться посторонним в секторе для прыжков;
    \item Прыжок выполняется только тогда, когда предыдущий прыгун покинул сектор для прыжка, а прыжковая яма подготовлена для следующего выполнения;
    \item Сектор для прыжков должен быть очищен от посторонних предметов.
\end{itemize}
\vspace*{14pt}

Спортивные и подвижные игры:
\begin{itemize}
    \item Все задания должны выполняться только после специальной разминки;
    \item Играть только в установленном месте;
    \item Одежда и обувь должны соответствовать направленности занятия;
    \item Все игровые задания выполняются с разрешения и под руководством преподава­теля.
\end{itemize}
\newpage

Занятия на тренажерах:
\begin{itemize}
    \item Приступать к выполнению упражнений на тренажерах только с разрешения пре­подавателя;
    \item Работать только на исправном оборудовании, при неисправностях тренажера со­общать преподавателю;
    \item Не разговаривать во время выполнения заданий;
    \item Регулировать вес и количество подходов только после консультаций с преподавателем;
    \item Не находиться вблизи тренажера, на котором выполняется задание;
    \item Соблюдать правила поведения для зала ОФП.
\end{itemize}
\vspace*{14pt}

\newpage
\begin{center}
    \textbf{Абсолютные противопоказания к занятиям физической культурой}
\end{center}
\vspace*{14pt}

В настоящее время установлены ограничения и абсолютные
противопоказания к занятиям физической культурой. В
большинстве случаев противопоказания определяются различными
заболеваниями внутренних органов.
\vspace*{14pt}

Перечень заболеваний и патологических состояний, препятствующих допуску к занятиям спортом:
\begin{itemize}
    \item Лихорадящие состояния, гнойные и воспалительные процессы,
    хронические заболевания в стадии обострения, острые инфекционные
    заболевания;
    \item Cердечно"=сосудистые заболевания;
    \item Хронические неспецифические заболевания легких с дыхательной
    недостаточностью \MakeUppercase{\romannumeral 2}"=\MakeUppercase{\romannumeral 3} степени;
    \item Заболевания крови;
    \item Последствия перенесенных черепно"=мозговых травм;
    \item Сосудистые заболевания;
    \item Последствия перенесенного острого нарушения мозгового
    кровообращения;
    \item Нервно"=мышечные заболевания;
    \item Рассеянный склероз;
    \item Злокачественные новообразования;
    \item Циррозы печени;
    \item Хронический гепатит, панкреарит;
    \item Близорукость высокой степени с изменением глаза;
    \item Сахарный диабет тяжелой формы;
    \item Остеохондроз позвоночника, осложненный грыжами дисков;
    \item Психические заболевания.
\end{itemize}

\newpage
\begin{center}
    \textbf{Заключение}
\end{center}
\vspace*{14pt}

Здоровье является неотъемлемой частью социального процветания, и потому
введение в здоровый образ жизни считается делом государственного масштаба.
Как известно, только здоровый человек способен наиболее эффективно создавать 
духовные и материальные ценности, генерировать новые идеи и творчески реализовывать их.
Быть здоровым "--- нормальное желание любого человека. Ведь именно здоровье "--- условие активности, успешности и долголетия человека.\\

Основная цель разделения на группы здоровья "--- выбор подходящих методов воспитания организма
в зависимости от индивидуальных особенностей и состояния здоровья человека с выбором того или иного режима,
уровня физической активности, нормативов и т.д.\\
Необходимо знать и правильно выполнять методические рекомендации для каждой категории учащихся.
Грамотно организованные занятия физической культуры "--- это приумножение своего собственного здоровья.
Быть здоровым "--- нормальное желание любого человека, ведь именно здоровье "--- условие активности, успешности и долголетия человека.
\newpage
\begin{center}
    \textbf{Список использованной литературы}
\end{center}
\vspace*{14pt}

\begin{enumerate}
    \item https://studopedia.ru/5\_42653\_kakie-sushchestvuyut-trebovaniya-bezopasnosti-pri-provedenii-zanyatiy-po-fizicheskoy-kulture.html
    \item http://www.grsmu.by/files/file/university/cafedry/fizicheskogo-vospitaniya-sporta/files/kontr\_norm\_spec\_2020\_2021.pdf
    \item https://infourok.ru/statya-harakteristika-medicinskih-grupp-2101611.html
    \item https://irkutsk.world-gym.com/uploads/image/ckeditor/1547111288\_Противопоказания.pdf
    \item https://express-med-service.ru/stati/meditsinskie-protivopokazaniya-k-zanyatiyam-sportom/
\end{enumerate}
\end{flushleft}